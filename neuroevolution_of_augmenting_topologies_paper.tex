% ######################################################################################################################

\documentclass[journal, a4paper]{IEEEtran}

\usepackage{graphicx}   % For graphics, photos, etc
\usepackage{hyperref}   % For URL and href
\usepackage{amsmath}    % For advanced mathematical formatting and symbols
\usepackage{blindtext}  % For placeholder text
\usepackage{listings}   % For code listings
\usepackage{color}      % For color

\definecolor{green}{rgb}{0, 0.66, 0}
\definecolor{red}{rgb}{1, 0, 0}
\definecolor{gray}{rgb}{0.5, 0.5, 0.5}
\definecolor{orange}{rgb}{1, 0.66, 0}
\definecolor{codebg}{rgb}{0.97, 0.97, 0.97}

\lstdefinestyle{c-style}{
  language={[ANSI]C},
  frame=single,
  backgroundcolor=\color{codebg},
  commentstyle=\itshape\color{green},
  keywordstyle=\color{blue},
  numberstyle=\tiny\color{gray},
  stringstyle=\color{orange},
  basicstyle=\fontsize{7}{7}\ttfamily,
  breakatwhitespace=false,
  breaklines=true,
  captionpos=b,
  keepspaces=true,
  numbers=left,
  numbersep=5pt,
  showspaces=false,
  showstringspaces=false,
  showtabs=false,
  tabsize=2
}

% ######################################################################################################################

\begin{document}

  \title{Neuroevolution of Augmenting Topologies}
  \author{Paul Pauls\\
          Advisor: Michael Adam}
  \markboth{Neuroevolution of Augmenting Topologies}{}
  \maketitle

\begin{abstract}
\blindtext
\end{abstract}

% ######################################################################################################################

\section{Introduction}

\IEEEPARstart{T}{his} shall be my introduction. And this shall be my citation \cite{cite01}.



% ######################################################################################################################

\section{Overview}

most commonly applied in artificial life, general game playing and evolutionary robotics.
main benefit is that Neuroevolution can be applied more widely than supervised learning algorithms [...] [as it] requires only a measure of a network's performance at a task

Neuroevolution is in competition with Gradient Descent. Around 2017 researchers at Uber stated they had found that simple structural Neuroevolution algorithms were competitive with sophisticated modern industry-standard gradient-descent deep learning algorithms, in part because Neuroevolution was found to be less likely to get stuck in local minima


\subsection{Classification of Neuroevolution Algorithms}
Conventional Neuroevolution:
evolve only the strength of the connection weights for a fixed network topology

TWEANNs (Topology and Weight Evolving Artificial Neural Network):
evolve both the topology of the network and its weights

Parallel or sequential Evolving:
A separate distinction can be made between methods that evolve the structure of ANNs in parallel to its parameters and those that develop them separately


\subsection{Genotypes and Direct/Indirect Encoding}
Evolutionary algorithms operate on genotypes. In neuroevolution, a genotype is mapped to a neural network phenotype that is evaluated on some task to derive its fitness.



% ######################################################################################################################

\section{Conclusion}

\blindtext



% ######################################################################################################################

\begin{thebibliography}{5}

  \bibitem{cite01}
    Example Cite, {\em Source}, Apr. 2019.
    \url{www.example.com}

\end{thebibliography}

\end{document}
